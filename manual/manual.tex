
\documentclass[11pt]{article}

\usepackage{calc}
\usepackage{graphics}
%\usepackage{latex8}
\usepackage{times}
\usepackage{epsf}
\usepackage{epsfig}
\usepackage{graphicx}
\usepackage{changebar}
\usepackage{portland}
\usepackage{lscape}

\setlength{\textheight}{8.50in}
\setlength{\textwidth}{6.5in}
\setlength{\topmargin}{-0.3in}
%\setlength{\leftmargin}{2.9in}
%\setlength{\rightmargin}{-2.9in}
\setlength{\oddsidemargin}{0in}
\setlength{\parindent}{0.5in}


\begin{document}

\title{
\includegraphics[height=1.5in]{v3vee.pdf}
\includegraphics[height=1.5in]{palacios.pdf} \\
\vspace{0.5in} 
Palacios Internal Developer Manual
}
\author{Jack Lange \hspace{0.5in} Lei Xia}

\maketitle


\section{Organization}


\section{Checking out Palacios}

Checkout or clone the devel branch of Palacios from the master
repository. You should have the read permission to these branches.


\section{Checking out Kitten}

hg clone /home/palacios/kitten

git clone /home/palacios/palacios

/opt/vmm-tools/bin/checkout\_branch devel


\section{Compiling Palacios}
cd palacios/build/


This will build Palacios as a library, libv3vee.a in the palacios/palacios/build/.


\section{Compiling Kitten}
\subsection{Configuration}
Kitten building can be configured by either text or graph configure interface, which is similar to the Linux kernel configure, By one of the following commands:

make xconfig
make config
make menuconfig

Make sure turn on the network device driver, networking, and input kernel command 'console=serial net=rtl8139'
\subsection{Compilation}

Build Palacios as a module for Kitten
In the first time, make sure to build Kitten before you building the Palacios as the module to kitten. 
Palacios now is built as a module of the Kitten. You can find the palacios.c and palacios.h in the kitten/palacios/. Enter the directory, build the palacios module.

cd kitten/palacios
make -C .. M=`pwd`
cp built-in.o ../modules/palacios-mod.o
Build Kitten
Go back to kitten root directory, and build the Kitten again.

make  isoimage

\section{Running Palacios/Kitten}
Run the whole stuff built above in Qemu using following command: 

/usr/local/qemu/bin/qemu-system-x86\_64 -smp 1 -m 1024 -serial file:./serial.out -cdrom ./arch/x86\_64/boot/image.iso  -net tap, ifname=tap0  < /dev/null


\section{networking}

\section{Configuring the development host's Qemu network}
Set up Tap interfaces:

/root/util/tap\_create tapX

Bridging tapX with eth1 will only work (work = send packet and also make packet visible on localhost) if the IP address is set correctly (correctly = match network it is connected to  e.g., network of eth1)  so bring up the network inside of the VM / QEMU as 10-net, and it should route through the eth1 rule and be visible both on the host and in the physical network


\subsection{Configuring Kitten}

How to set ip address in kitten:

Kitten ip address setting is in file drivers/net/ne2k/rtl8139.c, in the code below which is located in function rtl8139\_init.

  struct ip\_addr ipaddr = { htonl(0 | 10 << 24 | 0 << 16 | 2 << 8 | 16 << 0) }; 
  struct ip\_addr netmask = { htonl(0xffffff00) }; 
  struct ip\_addr gw = { htonl(0 | 10 << 24 | 0 << 16 | 2 << 8 | 2 << 0) };

This sets the ip address as 10.0.2.16, netmask 255.255.255.0 and gateway address 10.0.2.2, change it as you need.



\subsection{Running with networking}

\paragraph*{Tap Interface}
In which, the command line: 

-net tap, ifname=tap2

specifies Qemu to use the host's tap0 as its network interface, then Qemu can access the host's physical network.

\paragraph*{Redirection}

Also you can use the following command instead to redirect host's 9555 port to Qemu's 80 port.

-net user -net nic,model=rtl8139  -redir tcp:9555::80

In this case, you can access Qemu's 80 port in the host like:

telnet localhost 9555

Qemu has many options to build up a virtual or real networking. See http://www.h7.dion.ne.jp/~qemu-win/HowToNetwork-en.html for more information.




For more questions, talk to Jack or Lei.

\end{document}
