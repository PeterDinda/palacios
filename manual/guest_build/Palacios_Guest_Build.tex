\documentclass{article}[11pt]

\usepackage{amsmath}
\usepackage{amssymb}
\usepackage{verbatim}
\usepackage{epsf}
\usepackage{graphicx}

\def\colfigsize{\epsfxsize=5in}

\title{Building guest bootable image for the Palacios and Kitten}

\begin{document}
\maketitle

\section{Getting the stuffs used for building guest image}

In order to build the bootable guest iso image, we need to build Linux kernel image from its source, 
an initial file system containing a set of useful binary files and other necessary directories and files, 
which will be specified in the following text.

We will use a new directory for demonstration, 
the root directory for our following examples are all in the \verb+test/+ as root directory.

\begin{verbatim}
mkdir test/
cd test/
\end{verbatim}

There are a set of tools and sources that are useful for this building procedure. You can checkout these resources from our git repositories. To checkout them to your local directory, using commands:

\begin{verbatim}
git clone http://hornet.cs.northwestern.edu:9005/busybox
git clone http://hornet.cs.northwestern.edu:9005/initrd
git clone http://hornet.cs.northwestern.edu:9005/linux-2.6.30.y
\end{verbatim}

\section{Build the ramdisk filesystem}

Copy needed devices into the \verb+initramfs/dev/+ directory, 
for minimal support, you need console, ram, null, tty. 
(You probably need root privilege to copy and make the device files)

There is an already costumed initial ramdisk system made 
by Jack which you can leverage. The file is temporarily 
in \verb|/home/jarusl/initrd/disks/v3vee_initramfs.tar.gz|
in newskysaw machine.

\section{Configure and Install Busybox tools}

BusyBox is a software application released as Free software 
under the GNU GPL that provides many standard Unix tools. BusyBox 
combines tiny versions of many common UNIX utilities into a single 
small executable. For more details on Busybox, visiting http://busybox.net.

To configure BusyBox, in the \verb+busybox/+ directory, type

\begin{verbatim}
make meunconfig 
make xconfig (X version)
\end{verbatim}

You can add tools as you needed into the guest image. There are 
two specified configurations are required here:
First, in the "\verb|Build Options|", check the option 
"\verb|Build BusyBox as a static binary (no shared libs)|", 
as shown in figure \ref{fig:busyboxcf1}. Second, 
in the "\verb|Installation Options|", Change 
the "\verb|Busybox installation prefix|" to be the path 
of your "\verb|initramfs|" directory, as shown in figure \ref{fig:busyboxcf2}.
 
\begin{figure}\begin{center}\colfigsize\epsffile{busyboxConf1.eps}\end{center}\caption{BusyBox configuartion}\label{fig:busyboxcf1}\end{figure}

\begin{figure}\begin{center}\colfigsize\epsffile{busyboxConf2.eps}\end{center}\caption{BusyBox configuartion}\label{fig:busyboxcf2}\end{figure}
 
After you finish the configuration, save your configuration
and quit the window. Then, to make the Busybox tools, type 
\begin{verbatim}
make
\end{verbatim}
And then install the tools to your inital ramdisk directory

\begin{verbatim}
make install
\end{verbatim}

\section{Configure and Compile Linux Kernel}

Go to \verb|linux-2.6.30.y/| directory (or whatever your 
Linux kernel source directory). Type 

\begin{verbatim}
make menuconfig 
make xconfig (X version)
\end{verbatim}

To configure the kernel as your need. 
There is a customed configuration file "\verb|jrl-default-config|" 
which is configured with minimal kernel options (all of 
unnecessary stuffs are turned off to keep guest booting process faster).

For more on configuring and compiling Linux kernel image, check online. Here, one 
option for configuring the initial ramdisk file system directory (such 
as \verb|initrd/initramfs|) in the option 
"\verb|initial RAM filesystem and Ram Disk|" is needed, 
as shown in figure \ref{fig:linuxcf}.

\begin{figure}\begin{center}\colfigsize\epsffile{linuxConf.eps}\end{center}\caption{Linux Kernel configuartion}\label{fig:linuxcf}\end{figure}

\section{Configure guest devices}
Checkout the updated Palacios repository to \verb|palacios/|. 
(You can find the instruction of how to get the Palacios and
Kitten repository in http://www.v3vee.org/palacios/).

In the "\verb|palacios/utils/guest_creator|", Type
\begin{verbatim}
make
\end{verbatim}
You will get the guest configuration files creator "\verb|build_vm|".

The guest configuration file is using XML format. There is a sample configuration file: \verb|default.xml|. 
You can create your own guest configuration, after creating your file, type

\begin{verbatim}
./build_vm myconfig.xml -o guest.iso
\end{verbatim}

Where \verb+myconfig.xml+ is your guest configuration file, 
and \verb+guest.iso+ is the output image file that can be 
feed into kitten��s configuration (see next section).

\section{Configure and Make Palacios and Kitten}
\subsection*{Configure and build Palacios}

You can find the detail manual of getting and building Palacios and Kitten 
from scratch in the Palacios website (http://www.v3vee.org/palacios). 
Here we only give the specific requirements related to the booting guest procedure.

To configure Palacios, in Palacios root directory, such as \verb+palacios/+, type

\begin{verbatim}
make config
or
make xconfig
\end{verbatim}

To configure the components you want to build into Palacios. And then type 

\begin{verbatim}
make
or
make all
\end{verbatim}

To build Palacios static library, after build, you can find the Palacios library file \verb+libv3vee.a+ in root directory.

\subsection*{Configure and build Kitten}

To build Kitten, first configure it using the similar way as Palacios, go to \verb+kitten/+, type

\begin{verbatim}
make config
or
make xconfig
\end{verbatim}

Specifically, you have to enable Palacios VMM support in Kitten, 
and configure the root directory of Palacios and the 
path of the guest image file we just built in previous step, 
as show in figure \ref{fig:kittencf} and \ref{fig:kittencf2}.

\begin{figure}\begin{center}\colfigsize\epsffile{kittenConf1.eps}\end{center}\caption{Kitten configuartion}\label{fig:kittencf}\end{figure}

\begin{figure}\begin{center}\colfigsize\epsffile{kittenConf2.eps}\end{center}\caption{Kitten configuartion}\label{fig:kittencf2}\end{figure}
 
After configure, to build the Kitten, type

\begin{verbatim}
make isoimage
\end{verbatim}

to make the bootable ISO image file with guest OS, Palacios and Kitten.
The ISO file is located in \verb+kitten/arch/x86_64/boot/image.iso+.

Until here, you have successfully created a guest CD image file that can be 
booted on machine. You can boot the file on Qemu using following sample command

\begin{verbatim}
/opt/vmm-tools/qemu/bin/qemu-system-x86_64 \
        -smp 1 \
        -m 2047 \
        -serial file:./serial.out \
        -cdrom kitten/arch/x86_64/boot/image.iso \
        < /dev/null
\end{verbatim}

Here, we finally finish the entire procedure to 
build costumed guest and boot it on Palacios VMM. 
For more updated details, check Palacios 
website \verb+http://www.v3vee.org/palacios+ and 
Kitten website \verb+https://software.sandia.gov/trac/kitten+ regularly.


\end{document}