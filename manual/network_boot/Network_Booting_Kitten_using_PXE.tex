\documentclass[11pt]{article}
\usepackage{times}
\usepackage{fullpage}

\begin{document}

\title{\ \\ \LARGE\bf
Booting Palacios/Kitten Over the Network Using PXE
}

\author{Yuan Tang, Lei Xia}

\maketitle

PXE allows us to boot Kitten/Palacios (and a test guest) remotely from
a network server, which makes testing Palacios on physical machines
much easier.  The following gives instructions on how to set up the
PXE server and client.  Although we explain this specifically in the
context of Kitten/Palacios, PXE can be used to network boot most
operating systems.


In the following, we will assume there are two machines:
\begin{itemize}
\item {\em Server:}  This machine will serve Kitten/Palacios kernels over the network to client machines.
\item {\em Client:}  This machine will request Kitten/Palacios kernels from the server machine.
\end{itemize}

\section{How does it work?}

When the client machine starts up, PXE will first make a DHCP request
to retrieve a temporary IP configuration.   The server will respond to
this request.   Part of the response will indicate that the client
should reconnect back to the server (this time using TCP or UDP) to
retrieve more data using FTP.  The client will do so.  The server will
respond by sending a small bootloader, the kernel, and additional
files.   The client copies this content into relevant memory locations
and jumps to it.   The bootloader then boots the kernel.    In the
typical case with Palacios/Kitten, the kernel contains Kitten and
Palacios, while the additional file contains the init task, which in
turn has the guest image that's being tested embedded in it. 



\section{What will we boot?}

PXE can be used to build quite complex boot paths.   For testing
Palacios/Kitten, there are typically only two files involved, however:
\begin{itemize}
\item {\em  bzImage}:  This is the Kitten/Palacios kernel, typically
located\\
at \verb!your_kitten_path/arch/x86_64/boot/bzImage!.
\item {\em init\_task}:  This is the Kitten init task.   It also
contains the guest image that will be used for testing.  It is
typically located at \verb!your_kitten_path/init_task!. 
\end{itemize}

For booting, we copy these files into \verb./tftpboot. on the server. 


\section{Client requirements}

PXE is a service that is implemented in the client machine's BIOS or
on a network card's BIOS extensions.  Most modern machines, even
inexpensive desktops and laptops, support it but have it turned off by
default.  To turn it on, power cycle the client and enter the BIOS.
There may be an option to enable it here, probably under local
devices/networking.  If you can't find an option there, power cycle
again and wait for the BIOS bootstrap message about your network
card.  There should be an option to hit a key to go into its
configuration screen, in which you should be able to turn on PXE.  

\section{Server requirements}

\begin{itemize}
\item DHCP server. DHCP enables the client get an IP address, and
other configuration data automatically from the server.   
\item TFTP server.  After the client has an IP address, it will
connect back to the TFTP server it to download the kernel and other
files from the server.
\item Firewall rules that allow access from the client to the DHCP
server and TFTP server.  DHCP typically requires ports 67 and 68,
while TFTP typically uses port 69.
\item PXELINUX boot loader (\verb!pxelinux.0!) installed in
\verb./tftpboot..  You may also want to have \verb.memdisk. in the
same directory if you need to boot more ancient things. 
\end{itemize}


\section{DHCP setup on the server}

Install the DHCP service, and configure the file {\em
/etc/dhcpd.conf}.  You want to add the client's Ethernet MAC address
to config file, allow the client to get an IP address/config from
DHCP server at boot time.   Here is an example configuration:  
\begin{verbatim}
    option domain-name "cs.northwestern.edu";
    default-lease-time 600;
    max-lease-time 7200;

    allow booting;
    allow bootp;

    <-- Change below section according to your network configuration --->
    subnet 165.124.184.0 netmask 255.255.254.0 {
        range 165.124.184.90 165.124.184.95;
        option broadcast-address 165.124.184.255;
        option routers 165.124.184.1;
        option domain-name-servers 165.124.180.10;
    }

    group {
      <--- In which IP address the booting machine can find the booting image -->
      next-server 165.124.184.209;
      host v-test-intel-lab {
        <-- Replace below with the IP address assigned to the booting machine -->
        fixed-address 165.124.184.92;
        <--- Replace below with the MAC address of the booting machine -->
        hardware ethernet  00:1B:21:41:CF:4C;   
        filename "pxelinux.0";
      }
    } 
\end{verbatim}

Notice that you can readily add additional client machines by adding
new \verb.host. entries.

Now you can start the DHCP server.
\begin{verbatim}
[jdoe@newskysaw test]$ service dhcpd restart
\end{verbatim}


\section{TFTP Setup}

After installing the tftp service, enable it by editing the file {\em
/etc/xinetd.d/tftp} so that it looks similar to the following.
Typically you can just delete the line {\em disable = yes}.

\begin{verbatim}
    service tftp
    {
        socket_type             = dgram
        protocol                = udp
        wait                    = yes
        user                    = root
        server                  = /usr/sbin/in.tftpd
        server_args             = -s /tftpboot
        per_source              = 11
        cps                     = 100 2
        flags                   = IPv4
    }

\end{verbatim}

You can now the Kitten/Palacios kernel and init task to the tftp boot
directory, which typically is  {\em /tftpboot}:
\begin{verbatim}
[jdoe@newskysaw test]$ cp your_kitten_path/arch/x86_64/boot/bzImage  \
                              /tftpboot/bzImage
[jdoe@newskysaw test]$ cp your_kitten_path/init_task /tftpboot/init_task
\end{verbatim}

You will also want to be sure that you have copied \verb!pxelinux.0!
and \verb.memdisk. to \verb./tftpboot/.  

You can now start the TFTP service:
\begin{verbatim}
    service xinetd restart
\end{verbatim}


\section{PXE Configuration}

We will now set up the PXE boot configuration.  Create the
directory {\em pxelinux.cfg} in {\em /tftpboot}.  In {\em pxelinux.cfg},
create a new file with the filename as the client's MAC
address, prefaced with \verb.01-.:
\begin{verbatim}
[jdoe@newskysaw test]$ cd /tftpboot
[jdoe@newskysaw test]$ cd pxelinux.cfg
[jdoe@newskysaw test]$ touch 01-00-1b-21-41-cf-4c
\end{verbatim}
Edit this new confuration file ({\em 01-00-1b-21-41-cf-4c} here), to
set up a boot option for the machine.   For Kitten/Palacios testing,
the option typically looks like this:
\begin{verbatim}
    default kitten
    label kitten
        kernel bzImage
        append serial.baud=115200 console=serial initrd=init_task

    prompt 1 
    timeout 20
\end{verbatim}

Notice that your kernel bzImage file, and the init task file are both
referenced.  These are the copies that are /tftpboot.  You can have
multiple options in the configuration file, each with a different
label.   You can learn more about these configurations by reading
about PXELINUX.  

You can readily add additional machines simply by creating files
reflecting their MAC addresses.   This allows for one PXE server to
support numerous test machines and developers.

\section{Testing}

At this point things should be ready to go. Power on the client
machine, select network booting if prompted. PXE should then go
searching for the DHCP server, find our server, configure, and then
continue the boot process via TFTP.  Within seconds, you will see 
the Kitten/Palacios kernel start to boot.

If the client fails to connect to the PXE server, you should check the
firewall settings, making sure they allow the client's connection
requests.   Also, make sure the DHCP and TFTP services are actually
running. 

\section{Serial}

For testing, it is usually a good idea to connect the client machine
to some other machine via the serial port.   Palacios and Kitten
produce debugging and logging output via serial communication.   To
connect two machines, you will want a null modem RS232 cable.
Palacios and Kitten typically use 115200 bps communication.   If your
machine's serial port is /dev/ttyS0, then execute the following to see
the output:
\begin{verbatim}
stty -F /dev/ttyS0 115200
cat /dev/ttyS0
\end{verbatim}

Typical machines with a single port on the machine typically actually
have a dual UART.   If you have such a machine, you may want to try
\verb./dev/ttyS1. if \verb./dev/ttyS0. doesn't work.  

We have also experienced some challenges using SIIG multiport serial
interfaces, but these can usually be cleared up by configuring them
using:
\begin{verbatim}
setserial /dev/ttySn uart 16950 baud_base 115200
\end{verbatim}


\end{document}



